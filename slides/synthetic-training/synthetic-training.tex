\documentclass{bbawslides}

\usepackage[TS1, T1]{fontenc}
\usepackage{ifthen}

\usepackage[a4paper]{hyperref}
\rotateheaderstrue	%%-- otherwise seminar does strange things
\usepackage{url}

\usepackage[ngerman]{babel}
\usepackage[babel]{csquotes}
\usepackage[autoplay]{animate}
\usepackage{graphicx}
\usepackage{numprint}
\npthousandsep{~}\npthousandthpartsep{}\npdecimalsign{,}

\DeclareTextSymbol{\textlongs}{TS1}{115} 
\DeclareTextSymbolDefault{\textlongs}{TS1}


\begin{document}
\providecommand{\Title}{}


\begin{bbawtitle}[Praktische Probleme des OCR-Trainings mit synthetischen Daten]
  \vspace*{2em}%
  Kay-Michael Würzner\\[-.25em]%
  \textcolor{urlColor}{\texttt{{\small wuerzner@bbaw.de}}}
  \\[0.5em]
  {\small Gemeinsame Arbeit mit Ehud Alexander Avner und Matthias Boenig}
  \\[1.5em]
  {\footnotesize{%
    OCR-Entwicklerworkshop an der Berlin-Brandenburgischen Akademie der Wissenschaften\\%
    28. September 2017\\%
  }}
\end{bbawtitle}
\slideStyleFrame

\renewcommand{\footerText}{\tiny 28. September 2017, OCR-Entwicklerworkshop, BBAW}

%----------------------------------------------------------------------------------------------
% Outline
%----------------------------------------------------------------------------------------------
\begin{bbawslide}{Übersicht}
  \vspace*{7mm}%
  \centerslidestrue%
  \begin{itemize}
    \item Einleitung
    \begin{itemize}\small
      \item Was ist synthetisches Training?
      \item Wozu braucht man das?
    \end{itemize}
    \item Technische Aspekte
    \begin{itemize}\small
      \item Font-Rendering
      \item Unicode
      \item Schriftarten
      \item Volltexte
    \end{itemize}
    \item Experimente
    \begin{itemize}\small
      \item Hebräisch mit Nikkud
      \item Schreibmaschinenschrift
    \end{itemize}
  \end{itemize}
\end{bbawslide}

\begin{bbawpart}{\Large\bf Einleitung}
\end{bbawpart}

\begin{bbawslide}{Was ist synthetisches Training?}
  \vspace*{7mm}%
  \centerslidestrue%
  \begin{itemize}
    \item
  \end{itemize}
\end{bbawslide}

\begin{bbawpart}{\Large\bf Danke für Ihre Aufmerksamkeit!\\}
OCR-D-Team: Elisa Hermann, Maria Federbusch, Clemens Neudecker, Ajinkya Prabhune, \textbf{Matthias Boenig}\\
Mehr zu OCR: \url{https://www.zotero.org/groups/ocr-d}
\end{bbawpart}

\end{document}

%\begin{bbawpart}{\Large\bf Warum braucht man OCR?}
%\end{bbawpart}

%\begin{bbawslide}{Warum braucht man OCR?}
%  \vspace*{7mm}%
%  \centerslidestrue%
%  \begin{itemize}
%    \item
%  \end{itemize}
%\end{bbawslide}

%
% modelines
%

%%% Local Variables:
%%% mode: LaTeX
%%% coding: utf-8
%%% tab-width: 2
%%% indent-tabs-mode: nil
%%% End:

% vim: set ts=2 sw=2 expandtab :
